
\chapter{Differentiation}

\section{The Derivative of a Real Function}

\begin{definition} % Definition 5.1
    Let $f: [a, b] \to \R$. Then for any $x \in [a, b]$, define
    \[
        \phi(t) = \frac{f(t) - f(x)}{t - x}
    \]
    and
    \[
        f'(x) = \lim_{t \to x} \phi(t) = \lim_{t \to x} \frac{f(t) - f(x)}{t - x}.
    \]
    given the limit exists. This $f'$, whose domain is exactly the set of points $x$ where the above limit exists, is called the \textbf{derivative} of $f$.

    If $f'$ is defined at $x$, we say $f$ is \textbf{differentiable} at $x$. If $f$ is differentiable at every point of a set $E \subseteq [a, b]$, we say $f$ is differentiable on $E$. Extending the above limit definition with left-hand and right-hand limits, we get the left-hand and right-hand derivatives, respectively. Notice thus that $f'$ is not defined on the endpoints $a$ nor $b$.
\end{definition}

\begin{theorem} % Theorem 5.2
    Let $f : [a, b] \to \R$ be differentiable at $x \in (a, b)$. Then $f$ is continuous at $x$. 

    \begin{proof}
        As $t \to x$, we have
        \[
            f(t) - f(x) = \frac{f(t) - f(x)}{t - x} \cdot (t - x) \to f'(x) \cdot 0 = 0.
        \]
    \end{proof}

    Notice that the converse is not true: consider $f(x) = |x|$ on $[-1, 1]$.
\end{theorem}

\begin{theorem} % Theorem 5.3
    Suppose $f, g$ are defined on $[a, b]$ and differentiable at $x \in [a, b]$. Then $f + g, fg, f / g$ are differentiable at $x$, and
    \begin{enumerate}[(a)]
        \item $(f + g)'(x) = f'(x) + g'(x)$;
        \item $(fg)'(x) = f'(x) g(x) + f(x) g'(x)$;
        \item $(f/g)'(x) = \frac{f'(x) g(x) - f(x) g'(x)} {g^2(x)}$.
    \end{enumerate}
    
    \begin{proof}
        (a) follows from properties of limits. (b) follows from writing
        \[
            \frac{(fg)(t) - (fg)(x)}{t - x} = \frac{f(t) [g(t) - g(x)] + [f(t) - f(x)] g(x)}{t - x} \to f(x) g'(x) + f'(x) g(x).
        \]
        Similarly, (c) follows from writing
        \[
            \frac{(f/g)(t) - (f/g)(x)} = \frac{1}{g(t) g(x)} \left[g(x) \frac{f(t) - f(x)}{t - x} - f(x) \frac{g(t) - g(x)}{t - x} \right] \to \frac{f'(x) g(x) - f(x) g'(x)} {g^2(x)}.
        \]
    \end{proof}
\end{theorem}

It can be shown inductively that polynomials are differentiable, and that $(x^n)' = nx^{n-1}$. Also, 5.3(c) shows that rational functions are differentiable, except at points where the denominator is zero.

\begin{theorem} % Theorem 5.5
    Suppose $f: [a, b] \to \R$ is continuous on $[a, b]$ and differentiable at $x \in [a, b]$. Also, suppose $g$ is defined on an interval $I$ containing the range of $f$ and differentiable at $f(x)$. Then, $(g \circ f)'(x) = g'(f(x)) f'(x)$.
    
    \begin{proof}
        We have
        \[
            \frac{h(t) - h(x)}{t - x} = \frac{g(f(t)) - g(f(x))}{t - x} = \frac{g(f(t)) - g(f(x))}{f(t) - f(x)} \cdot \frac{f(t) - f(x)}{t - x} \to g'(f(x)) f'(x)
        \]
        where the first term converges since $f(t) \to f(x)$ as $f$ is continuous. 
    \end{proof}
\end{theorem}

\section{Mean Value Theorems}

\begin{definition} % Definition 5.7
    Let $f$ be a real-valued function on a metric space $X$. We say $f$ has a \textbf{local maximum} at a point $p \in X$ when there exists $\delta > 0$ such that $f(q) \le f(p)$ for all $q \in N_\delta(p)$. A \textbf{local minimum} is defined similarly.
\end{definition}

\begin{theorem} % Theorem 5.8
    Let $f: [a, b] \to \R$ with a local maximum at $x \in (a, b)$. If $f'(x)$ exists, then $f'(x) = 0$.

    \begin{proof}
        Define $f$ and $x$ as above, and suppose $f'(x)$ exists. If $f'(x) = c > 0$, then there must exist some $y > x$ sufficiently close (say, at most $\delta$ away) from $x$ such that $\phi(y) > 0$ but this implies $f(y) > f(x)$, a contradiction. The argument against $f'(x) < 0$ is symmetric.
    \end{proof}
\end{theorem}

\begin{theorem}[Cauchy's MVT] % Theorem 5.9
    If $f, g$ are continuous real functions on $[a, b]$ which are differentiable on $(a, b)$, then there is a point $x \in (a, b)$ at which
    \[
        [f(b) - f(a)] g'(x) = [g(b) - g(a)] f'(x).
    \]
    Notice that differentiability is not required at the endpoints.

    \begin{proof}
        Let $h(x) = [f(b) - f(a)] g(x) - [g(b) - g(a)] f(x)$. It suffices to show that $h'(x) = 0$ for some $x \in (a, b)$. Notice that $h(a) = h(b)$. If $h$ is constant on $[a, b]$, then $h'(x) = 0$ everywhere. Otherwise, $h$ admits a local extremum, and thus has derivative $0$ at that extremum, completing the proof.
    \end{proof}
\end{theorem}

\begin{corollary}[Mean Value Theorem] % Theorem 5.10
    If $f: [a, b] \to \R$ is differentiable on $(a, b)$, then 
    \[
        f(b) - f(a) = (b - a) f'(x)
    \]
    for some $x \in (a, b)$.

    \begin{proof}
        Take $g(x) = x$.
    \end{proof}
\end{corollary}

\begin{theorem} % Theorem 5.11
    Suppose $f$ is differentiable in $(a, b)$.
    \begin{enumerate}[(a)]
        \item If $f' \ge 0$ on $(a, b)$, then $f$ is monotonically non-decreasing.
        \item If $f' = 0$ on $(a, b)$, then $f$ is constant.
        \item If $f' \le 0$ on $(a, b)$, then $f$ is monotonically non-increasing.
    \end{enumerate}
n

    \begin{proof}
        For any pair $a \le x < y \le b$, there exists some $z \in (x, y)$ such that $f'(z) (y - x) = f(y) - f(x)$, so each result follows trivially.
    \end{proof}
\end{theorem}

\section{The Continuity of Derivatives}

We have seen that derivatives need not be continuous, but the following theorem states they still have the intermediate value property:

\begin{theorem}[Darboux property of derivatives] % Theorem 5.12
    Suppose $f: [a, b] \to \R$ is differentiable and $f'(a) < \lambda < f'(b)$. Then, there exists a point $x \in (a, b)$ such that $f'(x) = \lambda$.
    
    \begin{proof}
        Let $h(t) = f(t) - \lambda t$, and notice that $h'(a) = f'(a) - \lambda < 0$ and similarly $h'(b) > 0$. Then there exist $a < s < t < b$ such that $h(s) < h(a)$ and $h(t) < h(b)$. Then $h$ is continuous on the compact interval $[s, t]$ and thus attains a minimum at $u \in (s, t)$. Then $h'(u) = 0$ so $f'(u) = \lambda$ as required. 
    \end{proof}
\end{theorem}

\begin{corollary} 
    If $f$ is differentiable on $[a, b]$, then $f'$ has no simple discontinuities on $[a, b]$. (Discontinuities of the second kind as still possible)
\end{corollary}

\section{L'Hopital's Rule}

\begin{theorem}[L'Hopital's Rule] % Theorem 5.13
    Suppose $f, g$ are real and differentiable in $(a, b)$ and $g'(x) \ne 0$ for all $x \in (a, b)$ with $-\infty \le a < b \le +\infty$. Suppose $f'(x) / g'(x) \to A$ as $x \to a$.

    If $f(x) \to 0$ and $g(x) \to 0$ as $x \to a$, OR $g(x) \to +\infty$ as $x \to a$, then $f(x) / g(x) \to A$ as $x \to a$.

    \begin{proof}
        If $A < r < q < +\infty$, then we find $c_2$ such that $f(x) / g(x) < q$ for all $x \in (a, c_2)$. Repeating the argument to find $c_3$ such that $p < f(x) / g(x)$ for $x \in (a, c_3)$ completes the argument.

        Notice that since $f'(x) / g'(x) \to A$, we can find $c$ sufficiently close to $a$ such that $f'(x) / g'(x) < r$ for $x \in (a, c)$. Furthermore, 
        \[
            \frac{f(x) - f(y)}{g(x) - g(y)} = \frac{f'(t)}{g'(t)} < r
        \]
        for some $t \in (x, y) \subseteq (a, c)$ if $a < x < y < c$. If $f(x), g(x) \to 0$, then the previous expression becomes $f'(t) / g'(t) < r$ for all $t \in (a, c)$. Otherwise, if $g(x) \to +\infty$, then we can find $a < c_1 < y$ such that $g(x) > g(y)$ and $g(x) > 0$ whenever $x \in (a, c_1)$. Multiplying the previous expression by $[g(x) - g(y)] / g(x)$, we obtain
        \[
            \frac{f(x)}{g(x)} < r - r\frac{g(y)}{(g)x} + \frac{f(y)}{g(x)},\ x \in (a, c_1).
        \]
        If we let $x \to a$, then we can find $c_2$ such that $f(x) / g(x) \le r < q$ whenever $x \in (a, c_2)$, as required. This completes the proof.
    \end{proof}
\end{theorem}

\section{Derivatives of Higher Order}


