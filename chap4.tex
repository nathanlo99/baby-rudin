
\chapter{Continuity}

\section{Limits of Functions}

\begin{definition} % Definition 4.1
    Let $X, Y$ be metric spaces and $E \subseteq X$. Let $f$ map $E$ into $Y$, and $p$ be a limit point of $E$. We write $f(x) \to q$ as $x \to p$, or equivalently
    \[
        \lim_{x \to p}f(x) = q
    \]
    iff there is a point $q \in Y$ such that for every $\epsilon > 0$, there exists $\delta > 0$ such that $d_Y(f(x), q) < \epsilon$ for all points $x \in E$ such that $0 < d_X(x, p) < \delta$.

    Note that $p \in X$, but that $p$ need not be in $E$. Also, even if $p \in E$, we do not necessarily have $f(p) = \lim_{x \to p} f(x)$. This property is interesting and we will formalize it later.
\end{definition}

\begin{theorem} % Theorem 4.2
    Let $X, Y, E, f, p$ as in Definition 4.1. Then, $f(x) \to q$ when $x \to p$ iff $f(p_n) \to q$ for every sequence $\{p_n\}$ in $E \setminus \{p\}$ which converges to $p$.
\begin{proof}
    Suppose that $f(x) \to q$ as in Definition 4.1. Then, let $\epsilon > 0$ be arbitrary, and let $\delta$ such that Definition 4.1 holds. Then, if $\{p_n\}$ is some sequence converging to $p$, then eventually $d_X(p_n, p) < \delta$, so $d_Y(f(p_n), q) < \epsilon$ eventually. Thus $f(p_n) \to q$.

    Now suppose that $f(x) \not\to q$ as in Definition 4.1. Then, there exists some $\epsilon > 0$ such that for every $\delta > 0$, we can find some $x \in E$ such that $0 < d_X(x, p) < \delta$ but $d_Y(f(x), q) \ge \epsilon$. Let $\delta = 1/n$ for every $n$, giving us some sequence $x_n$ such that $d_Y(f(x_n), q) \ge \epsilon$. Thus $f(x_n) \not\to q$, proving the contrapositive.    
\end{proof}
\end{theorem}

\begin{corollary} % Corollary 4.3
    If $f$ has a limit at $p$, then this limit is unique.
\end{corollary}

\begin{definition} % Definition 4.4
    If $f, g$ are complex functions, then we denote $f + g$ to be the function such that $(f + g)(x) = f(x) + g(x)$. We define $f - g$, $fg$ and $f / g$ similarly. If $f(x) = c$ for all $x$, then we say $f$ is \textbf{constant}, and write $f = c$. If $f(x) \ge g(x)$ for all $x$, we sometimes write $f \ge g$.
\end{definition}

\begin{theorem} % Theorem 3.5
    Let $E \subseteq X$ a metric space, $p$ a limit point of $E$, and $f$ and $g$ complex functions on $E$. Furthermore, let $A = \lim_{x \to p} f(x)$ and $B = \lim_{x \to p} g(x)$. Then,
\begin{enumerate}
    \item $\lim_{x \to p} (f + g)(x) = A + B$.
    \item $\lim_{x \to p} (fg)(x) = AB$.
    \item $\lim_{x \to p} (f/g)(x) = A/B$ if $B \ne 0$.
    \item If $f$ and $g$ map into $\R^k$, then additionally $\lim_{x \to p} (f \cdot g)(x) = A \cdot B$.
\end{enumerate}

\begin{proof}
    This follows immediately from the analogous properties of sequences.
\end{proof}
\end{theorem}

\section{Continuous Functions}

\begin{definition} % Definition 4.5
    Suppose $X, Y$ are metric spaces, $E \subseteq X$, $p \in E$, and $f: E \to Y$. Then, $f$ is \textbf{continuous} at $p$ if for every $\epsilon > 0$, there exists $\delta > 0$ such that
    \[
        d_Y(f(x), f(p)) < \epsilon
    \]
    for all points $x \in E$ for which $d_X(x, p) < \delta$. If $f$ is continuous at every point in $E$, then $f$ is \textbf{continuous on $E$}.

    Notice that every function $f$ is continuous at isolated points of $p$, since we can always choose $\delta > 0$ sufficiently small such that $E \cap N_\delta(x) = \{p\}$ so $d_Y(f(x), f(p)) = 0 < \epsilon$.
\end{definition}

\begin{theorem} % Theorem 4.6
    Let everything as Definition 4.5. If $p$ is a limit point of $E$, then $f$ is continuous at $p$ iff $f(x) \to f(p)$ as $x \to p$.
\end{theorem}

\begin{theorem} % Theorem 4.7
    Suppose $X, Y, Z$ are metric spaces, $f: (E \subseteq X) \to Y$ and $g: f(E) \to Z$, and $h = g \circ f$. Then, if $f$ is continuous at $p \in E$ and $g$ is continuous at $f(p)$, then $h$ is continuous at $p$.

\begin{proof}
    Let $\epsilon > 0$. Since $g$ is continuous at $f(p)$, there exists some $\delta > 0$ such that $d_Z(g(f(x)), g(f(p))) < \epsilon$ for all $f(x) \in f(E)$ such that $0 < d_Y(f(x), f(p)) < \delta$. However, since $f$ is continuous, we can find $\delta' > 0$ such that $d_Y(f(x), f(p)) < \delta$ whenever $0 < d_X(x, p) < \delta'$. In other words, we've found $\delta' > 0$ such that $d_Z(h(x), h(p)) < \epsilon$ whenever $0 < d_X(x, p) < \delta'$, so $h$ is continuous at $p$ as required.
\end{proof}
\end{theorem}

\begin{theorem} % Theorem 4.8
    A mapping $f$ of a metric space $X$ into a metric space $Y$ is continuous on $X$ iff $f^{-1}(V)$ is open in $X$ for every open set $V$ in $Y$.

\begin{proof}
    Suppose $f$ is continuous on $X$. Then, if $y \in V$ for some open $V$, then there exists some neighbourhood $N_{\epsilon}(y)$ around $y$. By the definition of continuity, this admits $N_{\delta}(f^{-1}(y))$ in $f^{-1}(V)$ for positive $\delta$, so $f^{-1}(V)$ is open in $X$, so the forward direction is done since $V$ was chosen arbitrarily.

    Now for the contrapositive, suppose $f^{-1}(V)$ is open for every open subset $V$ of $Y$. Then, for any $x \in X$, for any $\epsilon > 0$, the set $V = N_\epsilon(f(x))$ is an open set in $Y$, so $f^{-1}(V)$ is open by assumption. Therefore, $f^{-1}(f(x)) = x$ is an interior point of $f^{-1}(V)$, so in particular, this gives us the $\delta > 0$ we need for continuity.
\end{proof}
\end{theorem}

\begin{corollary} % Corollary 4.8
    A mapping $f$ of a metric space $X$ into a metric space $Y$ is continuous iff $f^{-1}(C)$ is closed in $X$ for every closed set $C$ in $Y$.
\end{corollary}

\begin{theorem} % Theorem 4.9
    If $f, g$ are continuous complex functions on $X$, then $f+g$, $fg$ and $f/g$ are continuous on $X$.
\begin{proof}
    These follow directly from the corresponding theorems on sequences, and the equivalence of the sequential definition of continuity.
\end{proof}
\end{theorem}

\begin{theorem} % Theorem 4.10
    Let $f_1, f_2, \dotsc, f_k$ be real functions on $X$ and $\textbf{f}$ be the mapping of $X$ into $\R^k$ defined by
    \[
        \textbf{f}(x) = (f_1(x), f_2(x), \dotsc, f_k(x)).
    \]
    then $\textbf{f}$ is continuous iff each of the functions $f_1, \dotsc, f_k$ is continuous.

Furthermore, if $\textbf{f}, \textbf{g}: X \to \R^k$ are continuous, then $\textbf{f} + \textbf{g}$ and $\textbf{f} \cdot \textbf{g}$ are continuous.

\begin{proof}
    Part (a) follows from the fact that
    \[
    |f_i(x) - f_i(y)| \le |\textbf{x} - \textbf{y}| = \sqrt{ \sum_{j=1}^{k} |f_j(x) - f_j(y)|^2}, \forall i \in \{1, 2, \dotsc, k\},
    \]
    and part (b) follows from (a) and Theorem 4.9.
\end{proof}
\end{theorem}

Now, we show that a very large class of functions is continuous. Firstly, the component function $f((x_1, \dotsc, x_k)) = x_i$ is continuous for any $i \in \{1, 2, \dotsc, k\}$. This shows that polynomials in the components are continuous, and thus rational functions on $(x_i)$ are continuous. Finally, it can be seen that $f(\textbf{x}) = |\textbf{x}|$ is continuous.

\section{Continuity and Compactness}

\begin{definition} % Definition 4.13
    A mapping $f : E \to \R^k$ is \textbf{bounded} if there is a real number $M$ such that $|\textbf{f}(x)| \le M$ for all $x \in E$.
\end{definition}

\begin{theorem} % Theorem 4.14
    Suppose $f$ is a continuous mapping of a compact metric space $X$ into a metric space $Y$. Then $f(X)$ is compact.

\begin{proof}
    Let $\{G_\alpha\}$ be an open cover of $f(X)$. Then, since $f$ is continuous, each $f^{-1}(G_\alpha)$ is open, forming an open cover for $X$. Since $X$ is compact, 
    \[
        X \subseteq \bigcup_{i=1}^{n} f^{-1}(G_i)
    \]
    for some finite subset $\{G_i\}$ of $\{G_\alpha\}$. Then, since $f(f^{-1}(E)) \subseteq E$ for all $E \subseteq X$, we have $f(X) = \bigcup_{i=1}^{n} G_i$, so $f(X)$ is compact.
\end{proof}
\end{theorem}

\begin{theorem} % Theorem 4.15
    If $\textbf{f} : X \to \R^k$, then $\textbf{f}(X)$ is closed and bounded. Thus, $\textbf{f}$ is bounded.
\begin{proof}
    By Theorem 4.14, $\textbf{f}(X)$ is compact. Thus it is closed and bounded. In particular it is bounded.
\end{proof}
\end{theorem}

\begin{theorem}[Extreme Value Theorem] % Theorem 4.16
    Suppose $f$ is a continuous real function on a compact metric space $X$, and
    \[
        M = \sup_{p \in X} f(p),\quad m = \inf_{p \in X} f(p),
    \]
    then $f$ attains $M$ and $m$.

\begin{proof}
    From Theorem 4.14, $f(X)$ is compact and thus closed and bounded. Hence $f(X)$ contains $M$ and $m$, proving the result.
\end{proof}
\end{theorem}

\begin{theorem} % Theorem 4.17
    Suppose $f$ is a continuous bijection from a compact metric space $X$ onto a metric space $Y$. Then, $f^{-1}$ is a continuous mapping of $Y$ \underline{onto} $X$.
\begin{proof}
    To use Theorem 4.8, it suffices to show that $f(V)$ is open for every open set $V$ in $X$. But $V^c$ is closed and thus compact in $X$, and thus $f(V^c)$ is compact and thus closed in $Y$. Since $f$ is bijective, $f(V)$ is the complement is $f(V^c)$, thus $f(V)$ is open.
\end{proof}
\end{theorem}

\begin{definition} % Definition 4.18
    A mapping $f: X \to Y$ is \textbf{uniformly continuous} on $X$ if for every $\epsilon > 0$, there exists $\delta > 0$ such that
    \[
        d_Y(f(p), f(q)) < \epsilon, \forall p, q \in X : d_X(p, q) < \delta.
    \]
\end{definition}

\begin{theorem} % Theorem 4.19
    If $f$ is a continuous mapping from a \underline{compact} metric space $X$ into a metric space $Y$, then $f$ is uniformly continuous on $X$.
\begin{proof} 
    We prove the contrapositive: suppose $f$ is not uniformly continuous on $X$. Then, for some $\epsilon > 0$, we can find $p_n, q_n \in X$ such that $d_X(p_n, q_n) < 1/n$ but $d_Y(f(p_n), f(q_n)) > \epsilon$. Then, since $X$ is compact, we can find a subsequence $(p_{n_k})$ which converges, then take a subsequence $(q_{n_{k_m}})$ of $(q_{n_k})$ which converges, again since $X$ is compact. We relabel these subsequences $(r_n)$ and $(s_n)$. By the Squeeze Theorem, $r_n, s_n \to x_0$ for some $x_0 \in X$. However, $d_Y(f(r_n), f(s_n)) > \epsilon$, so $f$ cannot be continuous at $x_0$ by the sequential definition of continuity, completing the proof.
\end{proof}
\end{theorem}

\begin{theorem} % Theorem 4.20
    Let $E$ be a non-compact set in $\R$. Then:
    \begin{enumerate}
        \item There exists a continuous unbounded function on $E$.
        \item There exists a continuous and bounded function on $E$ which does not attain its maximum.
        \item If $E$ is bounded, there exists a continuous function on $E$ which is not uniformly continuous.
    \end{enumerate}

\begin{proof}
    For (a), if $E$ is unbounded, then $f(x) = x$ is continuous but unbounded. Otherwise, $E$ is not closed, so there exists a limit point $x_0$ of $E$ which is not in $E$. Then $f(x) = 1/(x-x_0)$ is continuous but unbounded (consider the sequence converging to $x_0$).

    For (b), if $f(x)$ is the function constructed in part (a), then $g(x) = -1/|f(x)|$ is bounded above by 0 and continuous, but never attains its maximum of 0.

    For (c), notice $E$ is thus not closed. Then $f(x)$ from part (a) is not uniformly continuous.
\end{proof}
\end{theorem}

\section{Continuity and Connectedness}

\begin{theorem} % Theorem 4.22
    If $f: X \to Y$ and $E$ is a connected subset of $X$, then $f(E)$ is connected.
\begin{proof}
    We prove the contrapositive: suppose $f(E) = A \cup B$ and $A, B$ are non-empty separated sets. Then, letting $A' = E \cap f^{-1}(A)$ and $B' = E \cap f^{-1}(B)$, we see that $E = A' \cup B'$.

Now 
\[
    \overline{A'} \cap B' \subseteq E \cap f^{-1}(\overline{A}) \cap f^{-1}(B) = \emptyset
\]
since $\overline{A} \cap B = \emptyset$, and repeating the argument for $A'$ and $\overline{B'}$ shows that $(A', B')$ is a separation of $E$, proving the contrapositive.
\end{proof}
\end{theorem}

\begin{theorem}[Intermediate Value Theorem] % Theorem 4.23
    Let $f$ be a continuous real function on the interval $[a, b]$. If $f(a) < c < f(b)$, then $f(x) = c$ for some $x \in (a, b)$.
\begin{proof}
If not, then $f([a, b]) \cap (f(a), c)$ and $f([a, b]) \cap (c, f(b))$ form a disconnection of $f([a, b])$, contradicting Theorem 4.22.
\end{proof}
\end{theorem}

\section{Discontinuities}

\begin{definition} % Definition 4.25
    If $f: X \to Y$ and $x \in X$ is such that $f$ is not continuous at $x$, we say $f$ is \textbf{discontinuous} at $x$, or that $f$ \textbf{has a discontinuity} at $x$.

    Let $f: (a, b) \to Y$. Let $x \in [a, b)$. We define $f(x+) = q$ if $f(t_n) \to q$ for any sequence $\{t_n\}$ in $(x, b)$ with $t_n \to x$. We define $f(x-)$ symmetrically. Clearly $\lim_{t \to x} f(t)$ exists iff $f(x-) = f(x+) = \lim_{t \to x} f(t)$. 
\end{definition}

\begin{definition} % Definition 4.26
    Let $f: (a, b) \to Y$. If $f$ is discontinuous at $x$, but $f(x-)$ and $f(x+)$ both exist, $f$ is said to have a \textbf{discontinuity of the first kind}, or a \textbf{simple discontinuity} at $x$. Otherwise, the discontinuity is said to be of the \textbf{second kind}.

    If $f$ has a simple disconinuity at $x$, either $f(x+) \ne f(x-)$, or $f(x+) = f(x-) \ne f(x)$.
\end{definition}

\section{Monotonic functions}

\begin{definition} % Definition 4.28
    Let $f: (a, b) \to \R$. Then $f$ is \textbf{monotonically increasing} on $(a, b)$ if $a < x < y < b$ implies $f(x) \le f(y)$. If instead $f(x) \ge f(y)$ with the same conditions, then $f$ is \textbf{monotonically decreasing}. 
\end{definition}

\begin{theorem} % Theorem 4.29
    Let $f$ be monotonically increasing on $(a, b)$. Then $f(x+)$ and $f(x-)$ exist at every point $x \in (a, b)$.
\begin{proof}
    Every sequence converging to $x$ from the left has a monotonically increasing subsequence. Then $f(x_n)$ is monotonically increasing and bounded above, say by $f(x+\delta)$. Thus $f(x_n)$ converges by the Monotone Convergence Theorem. The argument for $f(x+)$ is analogous.
\end{proof}
\end{theorem}

\begin{corollary}
Monotonic functions have no discontinuities of the second kind.
\end{corollary}

\begin{theorem} % Theorem 4.30
    Let $f$ be monotonic on $(a, b)$. Then the set of discontinuities of $f$ in $(a, b)$ is at most countable.
\begin{proof}
    We associate each discontinuity $x$ in $(a, b)$ with a rational number $r$ such that $f(x-) < r < f(x+)$, forming an injection from the set of discontinuities to $\Q$, proving the result.
\end{proof}
\end{theorem}

\section{Infinite Limits and Limits at Infinity}

\begin{definition} % Definition 4.32
    The set $(c, +\infty)$ is considered a neighbourhood of $+\infty$. Similarly for $(-\infty, c)$ being neighbourhoods of $-\infty$.
\end{definition}

\begin{definition} % Definition 4.33
    Let $f: (E \subseteq \R) \to Y$. We say $f(t) \to y$ as $t \to x$, where $A, x \in \overline{\R}$, if for every neighbourhood $U$ of $y$, there is a neighbourhood $V$ of $x$ such that $V \cap E$ is non-empty, and $f(t) \in U$ for all $t \in V \cap E \setminus \{x\}$. Limits of functions still combine in the same ways. 
\end{definition}

\section{Exercises}
\begin{enumerate}
\item % Exercise 4.1
Suppose $f: \R \to \R$ with $\lim_{h \to 0}[f(x+h) - f(x-h)] = 0$ for all $x \in \R$. Is $f$ continuous?
\begin{proof}
No, consider the function $f$ which is $1$ at $0$ and $0$ everywhere else. Then $f$ satisfies the condition everywhere but is discontinuous at $x = 0$.
\end{proof}

\item % Exercise 4.2
Let $f: X \to Y$ be continuous. Prove that
\[
    f(\overline{E}) \subseteq \overline{f(E)}
\]
for every $E \subseteq X$. Show that this inequality can be proper.

\begin{proof}
    Let $E$ be arbitrary and $x \in \overline{E}$. If $x \in E$, then clearly $f(x) \in f(E) \subseteq \overline{f(E)}$. Otherwise, $x$ is a limit point of $E$, so some sequence $(x_n)$ in $E$ converges to $x$, so since $f$ is continuous, the sequence $f(x_n) \to f(x)$, and thus $f(x) \in \overline{f(E)}$, as required.

    Let $f(x) = 1$ for all $x < 1$ and $f(x) = 1/x$ if $x > 1$. Then $f$ is continuous on $\R$, but $0 \in \overline{f(\R)}$ but is not in $f(\R)$, so the inclusion can be proper.
\end{proof}

\item % Exercise 4.3
Let $f: X \to \R$ be continuous. Let $Z(f)$ be the kernel of $f$. Prove that $Z(f)$ is closed.
\begin{proof}
    Let $x_n \to x$ be a convergent sequence in $Z(f)$. Then $f(x_n) = 0$ for all $n$, and thus $f(x_n) \to 0$. Since $f$ is continuous, this implies that $f(x) = 0$, and thus that $x \in Z(f)$, completing the proof.
\end{proof}

\item % Exercise 4.4
Let $f, g: X \to Y$ be continuous, and $E$ be dense in $X$. Prove that $f(E)$ is dense in $f(X)$. If $g(p) = f(p)$ for $p \in E$, prove that $g(p) = f(p)$ for all $p \in X$.

\begin{proof}
    Let $V \subseteq f(X)$ be open. Since $f$ is continuous, $f^{-1}(V)$ is open, and since $E$ is dense, we can find $p \in E \cap f^{-1}(V)$, which corresponds to $p \in E$ such that $f(p) \in V$, so in particular, $f(p)$ is an element of $f(E)$ in $V$. Since $V$ was arbitrary, $f(E)$ is dense in $f(X)$. 

    Notice that $E$ is a dense subset of $Z = Z(g-f)$. Thus, $\overline{E} = X \subseteq \overline{Z} \subseteq X$, so in particular, $X = Z$ since $Z$ is closed from the previous Exercise, completing the proof.
\end{proof}

\item % Exercise 4.5
    Let $f: (E \subseteq \R) \to \R$ be continuous, and $E$ be closed. Prove that there exist continuous functions $g: \R \to \R$ such that $g = f$ on $E$. These are called \textbf{continuous extensions} of $f$ from $E$ to $\R$. Show that the result becomes false if $E$ is not closed, and extend the result to vector-valued functions.

\begin{proof}
    We know that $E^c$ is open and thus a countable union of disjoint open segments of $\R$. For each of these segments $(a, b)$, assign $g(at + b(1-t)) = tf(a) + (1-t)f(b)$, and $g(x) = f(x)$ on $E$. Then $g = f$ on $E$, and $g$ is continuous at the endpoints $a$ and $b$, as required.
    
    Now, if $E = \{\pm 1/n: n \in \N^+\}$. Then let $f(1/n) = 1+1/n$ and $f(-1/n) = -1+1/n$. Then the sequences $f(1/n) \to 1$ and $f(-1/n) \to -1$ while both $\pm 1/n \to 0$. Thus any $g$ which agrees with $f$ on $E$ must be discontinuous at $0$, and thus no such continuous extension of $f$ exists.
\end{proof}

\item % Exercise 4.6
If $f$ is defined on $E$, the \textbf{graph} of $f$ is the set of points $(x, f(x))$ for $x \in E$. Suppose $E$ is compact. Prove that $f$ is continuous on $E$ iff its graph is compact.

\begin{proof}
    First, suppose $f$ is continuous. Let $\{G_\alpha\}$ be an open cover of $(E, f(E))$. Then since $f$ is continuous, $E$ and $f(E)$ are both compact, so we can choose elements $\{G_n\}$ such that the restrictions of the $G_n$ to $X$ cover $E$, and similarly $G_m$ to cover $f(E)$. These two sets combined give a finite cover for the graph of $f$.

    Now suppose the graph of $f$ is compact. Then \TODO
\end{proof}

\item % Exercise 4.7
If $E \subseteq X$ and $f$ is defined on $X$, the \textbf{restriction} of $f$ to $E$ is the function whose domain is $E$, such that $g(p) = f(p)$ for $p \in E$. Define $f, g: \R^2 \to \R$ such that $f(0, 0) = g(0, 0) = 0$, $f(x, y) = xy^2/(x^2 + y^4)$, $g(x, y) = xy^2/(x^2 + y^6)$ otherwise. Prove that $f$ is bounded on $\R^2$, and that $g$ is unbounded in every neighbourhood of $(0, 0)$, and that $f$ is not continuous at $( 0)$; but nevertheless that the restrictions of both $f$ and $g$ to every straight line in $\R^2$ are continuous!

\begin{proof}
    Notice that $\frac{ab}{a^2 + b^2} \le \frac{1}{2}$ since $(a - b)^2 = a^2 - 2ab + b^2 \ge 0$ for any $a, b$. Thus $f$ is bounded. Now, we have $g(1/n^3, 1/n) = n/2$ so $g$ is unbounded in any neighbourhood of $(0, 0)$. Finally, $f(1/n^2, 1/n) = 1/2$ for any $n$, so $f$ is discontinuous at $(0, 0)$. 

    However, if $x = cy$, $f(cy, y) = \frac{cy^3}{c^2y^2 + y^4} = \frac{cy}{c^2 + y^2} \to 0$ as $y \to 0$, so $f$ is continuous on the line $x = cy$. On any other line not passing through $(0, 0)$, the restriction is purely within the domain of the continuous definition. Similarly, $g(cy, y) = \frac{cy}{c^2 + y^4} \to 0$ as $y \to 0$, so $g$ is continuous on any straight line.
\end{proof}

\item % Exercise 4.8
Let $f: E \to \R$ be uniformly continuous on a bounded set $E \subseteq \R$. 
\begin{proof}
    

\end{proof}

\item % Exercise 4.9
Show that the requirement in the definition of uniform continuity can be rephrased as follows, in terms of diameters of sets: To every $\epsilon > 0$, there exists $\delta > 0$ such that $\diam f(E) < \epsilon$ for all $E \subseteq X$ with $\diam E < \delta$.

\item % Exercise 4.10
Complete the details of the following alternative proof of Theorem 4.19.
\begin{proof}
Plot twist: I accidentally wrote this proof as I went along proving while copying over the notes.
\end{proof}

\item % Exercise 4.11
Suppose $f: X \to Y$ is uniformly continuous, and let $\{x_n\}$ be Cauchy in $X$. Prove that $\{f(x_n)\}$ is Cauchy in $Y$. Use this result to give an alternative proof of the theorem stated in Exercise 13.

\item % Exercise 4.12
A uniformly continuous function of a uniformly continuous function is uniformly continuous. State this more precisely and prove it.

\item % Exercise 4.13
Let $E$ be a dense subset of a metric space $X$, and let $f: E \to \R$ be uniformly continuous. Prove that $f$ has a continuous extension from $E$ to $X$. Could the range space $\R$ be replaced by $\R^k$? By any compact metric space? By any complete metric space? By \underline{any} metric space?

\item % Exercise 4.14
Let $I$ be the closed unit interval, and $f: I \to I$ be continuous. Prove that $f(x) = x$ for at least one $x \in I$.
\begin{proof}
    If $f(0) = 0$ or $f(1) = 1$, we are done. Otherwise, $f(0) > 0$ and $f(1) < 1$, so the function $g(x) = f(x) - x$ is positive at $0$ and negative at $1$. By the Intermediate Value Theorem, $g(x) = f(x) - x = 0$ so $f(x) = x$ for some $x \in (0, 1) \subseteq I$, as required.
\end{proof}

\item % Exercise 4.15
A mapping $f: X \to Y$ is \textbf{open} if $f(V)$ is open in $Y$ whenever $V$ is open in $X$. Prove that every continuous open mapping $f: \R \to \R$ is monotonic.

\item % Exercise 4.16
Let $[x]$ denote the floor of $x$, and $(x) = x - [x]$. What discontinuities do $[x]$ and $(x)$ have?
\begin{proof}
They are both discontinuous at the integers.
\end{proof}

\item % Exercise 4.17
Let $f: (a, b) \to \R$. Prove that the set of points at which $f$ has a simple discontinuity is at most countable.

\item % Exercise 4.18
Every rational $x$ can be written as $x = m/n$ with $n > 0$ and $(m, n) = 1$. When $x = 0$, take $n = 1$. Then, consider $f(x) = 1/n$ if $x$ is rational, and 0 otherwise. Prove that $f$ is continuous at every irrational point, and that $f$ has a simple discontinuity at every rational point.

\item % Exercise 4.19
Suppose $f: \R \to \R$ has the intermediate value property, and that for every rational $r$, $f^{-1}(r)$ is closed. Prove that $f$ is continuous.

\item % Exercise 4.20
If $E$ is a non-empty subset of a metric space $X$, define the distance from $x \in X$ to $E$ by $\rho_E(x) = \inf_{z \in E} d(x, z)$.
\begin{enumerate}
\item Prove that $\rho_E(x) = 0$ iff $x \in \overline{E}$.
\item Prove that $\rho_E$ is a uniformly continuous function on $X$ by showing that
\[
    |\rho_E(x) - \rho_E(y)| \le d(x, y), \forall x, y \in X
\]
\end{enumerate}

\item % Exercise 4.21
Suppose $K, F$ are disjoint sets in $X$, where $K$ is compact and $F$ is closed. Prove that there exists $\delta > 0$ such that $d(p, q) > \delta$ if $p \in K$ and $q \in F$. Show that the conclusion may fail for two disjoint closed sets if neither is compact.

\item % Exercise 4.22
Let $A, B$ be disjoint non-empty closed sets in a metric space $X$, and define
\[
    f(p) = \frac{\rho_A(p)}{\rho_A(p) + \rho_B(p)}.
\]
Show that $f$ is a continuous function on $X$ whose range lies in $[0, 1]$, and that $f(p) = 0$ precisely on $A$ and $f(p) = 1$ precisely on $B$. This establishes a converge of Exercise 3: Every closed set $A \subseteq X$ is $Z(f)$ for some continuous real $f$ on $X$. Setting 
\[
    V = f^{-1}([0, 1/2)) \text{ and } W = f^{-1}((1/2, 1]),
\]
show that $V, W$ are open and disjoint, and that $A \subseteq V$, $B \subseteq W$, and thus that pairs of disjoint closed sets in a metric space can be covered by pairs of disjoint open sets. This property of metric spaces is called \textbf{normality}.

\item % Exercise 4.23
A function $f: (a, b) \to \R$ is \textbf{convex} iff
\[
    f(\lambda x + (1 - \lambda) y) \le \lambda f(x) + (1 - \lambda) f(y)
\]
for $x, y \in (a, b)$ and $\lambda \in (0, 1)$. Prove that every convex function is continuous. Prove that every increasing convex function of a convex function is convex.

If $f$ is convex in $(a, b)$ and $a < s < t < u < b$, show that
\[
    \frac{f(t) - f(s)}{t - s} \le \frac{f(u) - f(s)}{u - s} \le \frac{f(u) - f(t)}{u - t}.
\]

\item % Exercise 4.24
Let $f: (a, b) \to \R$ be continuous such that
\[
    f\left(\frac{x+y}{2}\right) \le \frac{f(x) + f(y)}{2}, \forall x, y \in (a, b).
\]
Show that $f$ is convex.

\item % Exercise 4.25
If $A, B \subseteq \R^k$, define $A + B = \{x + y: x \in A, y \in B\}$.
\begin{enumerate}
\item If $K$ is compact and $C$ is closed in $\R^k$, prove that $K + C$ is closed.
\item Let $\alpha$ be irrational. Let $C_1 = \Z$ and $C_2 = \alpha\Z$. Show that $C_1, C_2$ are closed subsets of $\R$ whose sum $C_1 + C_2$ is \textbf{not} closed, by showing that $C_1 + C_2$ is a countable dense subset of $\R$.
\end{enumerate}

\item % Exercise 4.26
Suppose $X, Y, Z$ are metric spaces such that $Y$ is compact. Let $f: X \to Y$, and $g: Y \to Z$ be a continuous bijection, and let $h = g \circ f$.

Prove that $f$ is uniformly continuous if $h$ is uniformly continuous. Prove also that $f$ is continuous if $h$ is continuous. 

Show, that the compactness of $Y$ cannot be omitted from the hypotheses, even when both $X$ and $Z$ are compact.
\end{enumerate}
