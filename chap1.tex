\chapter{The Real and Complex Number Systems}

\section{Ordered Sets}
We realize, considering the solutions to $x^2 = 2$, that the set of rational numbers $\Q$ is incomplete. We start by defining some terms:
\begin{definition}
An \textbf{order} $<$ on $S$ is a relation such that 
\begin{enumerate}
\item If $x, y \in S$, then exactly one of $x < y$, $x = y$, or $x > y$ is true.
\item If $x, y, z \in S$ with $x < y$ and $y < z$, then $x < z$. 
\end{enumerate}

A set $S$ is \textbf{ordered} if it admits an order. For example, $\Q$ is an order if we take $x < y \iff y - x$ is positive. 

If $S$ is an ordered set and $E \subseteq S$, we call $\beta \in S$ an \textbf{upper bound} for $E$ if $\beta \ge x$ for all $x \in S$. If such a $\beta$ exists, we say $E$ is \textbf{bounded above} (by $\beta$). We define lower bounds similarly. 

If $S$ is an ordered set and $E \subseteq S$ is bounded above, and there exists $\alpha \in S$ such that $\alpha$ is an upper bound for $E$ and if $\gamma < \alpha$ then $\gamma$ is not an upper bound for $E$, then we call $\alpha$ a \textbf{least upper bound} for $E$. We write $\alpha = \sup E$.

Similarly, we can define a \textbf{greatest lower bound} for $E$, which we write $\alpha = \inf E$. 

An ordered set $S$ is said to have the \textbf{least upper bound property} if every non-empty subset $E \subseteq S$ which is bounded above admits a least upper bound. We define the \textbf{greatest lower bound property} similarly. The following theorem demonstrates that these properties are in fact equivalent for all ordered sets.
\end{definition}


\begin{theorem}
If $S$ is an ordered set with the least upper bound property, and $\emptyset \ne B \subseteq S$ is bounded below, then $\inf B \in S$. 
\begin{proof}
Let $L$ be the set of lower bounds for $B$ in $S$. Since $B$ is bounded below, $L$ is non-empty; and a subset of $S$ by definition. Since $S$ has the least upper bound property, there exists some $\alpha = \sup L \in S$. We claim that $\alpha = \inf B$ and in particular, $\inf B \in S$.

To show this, it suffices to show that $\alpha$ is a lower bound for $B$, and that if $\beta > \alpha$ in $S$, then $\beta$ is not a lower bound for $B$. If $\alpha$ were not a lower bound for $B$, there must be some $x \in B$ with $x < \alpha$. However this would imply that $x$ is an upper bound for $L$ and thus $\alpha \ge \beta$, a contradiction. 

If $\beta > \alpha$ and $\beta$ were a lower bound for $B$ (so $\beta \in L$), then since $\alpha = \sup L$, we would have $\beta \le \alpha$, a contradiction.

Thus, $\alpha = \inf B \in S$. In particular, since this holds for all non-empty $B \subseteq S$ bounded below, $S$ has the greatest lower bound property.
\end{proof}
\end{theorem}

\section{Fields}
On to defining the notion of fields!

\begin{definition}
A field $F$ is a set equipped with two operations, $+$ and $\times$ with the following properties:
\begin{itemize}
\item $F$ is a commutative group under addition, with additive identity $0$. We denote the additive inverse of $x$ to be $-x$.
\item $F \setminus \{0\}$ is a commutative group under multiplication, with additive identity $1$. We denote the multiplicative inverse of $x$ to be $1/x$. 
\item $x \times (y + z) = x \times z + y \times z$ for all $x, y, z \in F$. 
\end{itemize}
\end{definition}

\begin{proposition}
If $G$ is a group with operation $\cdot$, then for $x, y, z \in G$,
\begin{enumerate}[(a)]
\item $x \cdot y = x \cdot z \implies y = z$
\item $x \cdot y = x \implies y = 1$
\item $x \cdot y = 1 \implies y = x^{-1}$
\item $(x^{-1})^{-1} = x$
\end{enumerate}
\begin{proof}
Statement (a) follows from multiplying $x^{-1}$ to both sides. (b) follows by letting $z = 1$ in (a); (c) by taking $z = x^{-1}$; and (d) follows since $x^{-1} \cdot x = 1$ and by replacing $x$ with $x^{-1}$ in (c). 
\end{proof}
Notice that these properties can be applied both to the additive and multiplicative groups in a field $F$. 
\end{proposition}

In addition to the above group properties, the distributive law leads to the following properties of fields:

\begin{proposition}
In a field $F$, for any $x, y \in F$, we have
\begin{enumerate}[(a)]
\item $0 \times x = 0$.
\item If $x \ne 0$ and $y \ne 0$, then $x \times y \ne 0$.
\item $(-x) \times y = - (x \times y) = x \times (-y)$.
\item $(-x) \times (-y) = xy$.
\end{enumerate}

\begin{proof}
For (a), notice that since $0 + 0 = 0$, we have $0 \times x = (0 + 0) \times x = 0 \times x + 0 \times x$, and the result follows from Proposition 1(b) on the additive group.

For (b), notice that if $x \times y = 0$ in this case, multiplying both sides of the equality by $y^{-1}x^{-1}$ would produce $1 = 0$, a contradiction.

For (c), notice that $0 = (x + (-x)) \times y = x \times y + (-x) \times y$, and the rest follows from Proposition 1(c). The other side follows similarly.

Statement (d) follows from (c) with $-y$ in place of $y$ and using Proposition 1(d). 
\end{proof}
\end{proposition}

Now that we've sufficiently introduced ordered sets and fields, we wonder what would happen if we combined them...

\begin{definition}
An \textbf{ordered field} $F$ is a field which is also an ordered set, with the following properties for all $x, y, z \in F$:
\begin{enumerate}
\item $x + y < x + z$ if $y < z$.
\item $x \times y > 0$ if $x > 0$ and $y > 0$
\end{enumerate}

If $x > 0$, we say $x$ is \textbf{positive}, and we say $x$ is \textbf{negative} if $x < 0$. For example, $\Q$ is an ordered field with the regular notions of $+$, $\times$ and $<$. 
\end{definition}

The following properties hold for ordered fields:
\begin{proposition}
For $x, y, z \in F$ for some ordered field $F$, we have
\begin{enumerate}
\item If $x > 0$, then $-x < 0$.
\item If $x > 0$ and $y < z$ then $xy < xz$.
\item If $x < 0$ and $y < z$ then $xy > xz$.
\item If $x \ne 0$ then $x^2 > 0$. In particular, $1 > 0$.
\item If $0 < x < y$ then $0 < y^{-1} < x^{-1}$. 
\end{enumerate}
\begin{proof}
For (a), if $x > 0$, then $0 = x + (-x) > 0 + (-x) = -x$. The other direction follows similarly. 

For (b), notice that since $y < z$, we have $z - y > 0$ and thus $x (z - y) > 0$. Then, notice that $xz = x (z - y) + xy > xy$. 

For (c), notice that $-x > 0$ by (a) so $-(xy) = (-x)y < (-x)z = -(xz)$. The result follows by adding $xy + xz$ to both sides.

For (d), if $x > 0$, then $x^2 = xx > x \times 0 > 0$, otherwise $-x > 0$ and the argument is analogous since $(-x)^2 = -(-x^2) = x^2$. In particular, $1 = 1 \times 1 = 1^2 > 0$. 

For (e), notice that since $x > 0$, we must have $x^{-1} > 0$ otherwise $1 = x \times x^{-1} < 0$, contradicting (d). A similar argument holds for $y^{-1}$, and thus $x^{-1}y^{-1} > 0$. The result follows by multiplying the inequality by the positive value $x^{-1}y^{-1}$. 
\end{proof}
\end{proposition}

\section{The Real Field}
It is tempting to combine the definitions we have introduced so far: we can imagine some ordered field which has the least upper bound property, containing the rational numbers $\Q$. In fact, this is the familiar set of real numbers, $\R$, as shown in the following theorem:

\begin{theorem}[Existence (and construction) of $\R$]
There exists an ordered field $\R$ with the least upper bound property, and in fact, it contains $\Q$ as a subfield. Specifically, this means that $\Q \subseteq \R$, and that the field operations in $\R$ coincide with the field operations in $\Q$. 
\begin{proof}
This proof is presented in the Appendix.
\end{proof}
\end{theorem}

\begin{theorem}
From here, we can derive the following properties in $\R$:
\begin{enumerate}[(a)]
\item If $x, y \in \R$ and $x > 0$, then there is a positive integer $n$ such that $nx > y$. 
\item If $x < y$ in $\R$, there exists $q \in \Q$ with $x < q < y$.
\item For every $x > 0$ in $\R$, there exists one and only one positive $y \in \R$ such that $y^n = x$, for every positive integer $n$.
\item If $a, b > 0$ and $n > 0$, then $(ab)^{1/n} = a^{1/n} b^{1/n}$.
\end{enumerate}
\begin{proof}
For (a), let $A = \{nx: n \in \N\}$. If (a) were true, $A$ would be bounded above, and clearly non-empty, and thus admits a least upper bound $\alpha = \sup A$ in $\R$. However, $\alpha - x < \alpha$ is not an upper bound for $A$, so there exists some $\alpha - x < mx \le \alpha$ in $A$. But then $\alpha < (m + 1)x \in A$, a contradiction.

For (b), notice that $y - x > 0$ so from (a), there exists $n \in \N$ such that $n(y - x) > 1$. Apply (a) again to find $m_1 > nx$ and $m_2 > -nx$ so that $-m_2 < nx < m_1$. Thus, there exists $m \in \N$ such that $m - 1 \le nx < m$. Adding $n(y - x) > 1$ to both sides of $nx \ge m - 1$ yields $ny > m$ and so dividing both sides of $nx < m < ny$ by $n > 0$ yields the result.

For (c), notice first of all that the function $f(x) = x^n$ is monotonically increasing on the positive real numbers. To see this, factor
\[
	b^n - a^n = (b - a)(b^{n-1} + b^{n-2}a + \dotsb + b^2 a^{n-3} + ba^{n-2} + a^{n-1}) > n(b-a)a^{n-1} > 0
\]
and similarly we can show $b^n - a^n < n(b-a)b^{n-1}$. These inequalities will also be helpful later. The fact that $n$-th positive real roots are unique follows immediately from the fact that $x^n$ is strictly increasing. 

Let $A = \{y > 0: y^n < x\}$. Then, $A$ is bounded above by $\max(1, x)$ and contains $\frac{x}{1+x}$. Since $\R$ satisfies the least upper bound property, $\alpha = \sup A$ exists. We claim that $\alpha^n = x$. 

Suppose otherwise. If $\alpha^n < x$, then for
\[
	0 < t < \min\left(1, \frac{x-\alpha^n}{n(\alpha + 1)^{n-1}}\right), \text{ we have }(\alpha + t)^n - \alpha^n < nt(\alpha + t)^n < nt(\alpha + 1)^n < x - \alpha^n
\]
so $(\alpha + t)^n < x$, contradicting the fact that $\alpha$ is an upper bound for $A$. 

If $\alpha^n > x$, then for
\[
	0 < t = \min\left(1, \frac{\alpha^n - x}{n\alpha^{n - 1}}\right), \text{ we have } \alpha^n - (\alpha - t)^n < nt\alpha^{n-1} = \alpha^n - x
\]
so $(\alpha - t)^n > x > z^n$ for each $z \in A$, so in particular $\alpha - t$ is a smaller upper bound than $\alpha$ for $A$, contradicting the fact that $\alpha$ is the \textbf{least} upper bound for $A$. 

For (d), notice that $a^{1/n}b^{1/n} > 0$ since both terms are positive from (c), and that $(a^{1/n}b^{1/n})^n = ab$ since multiplication is commutative, so from (c), the $n$-th positive root is unique, $(ab)^{1/n} = a^{1/n}b^{1/n}$. 
\end{proof}
\end{theorem}

\begin{definition}
Let $x \in \R_{>0}$. Let $n_0$ be the largest integer such that $n_0 \le x$. Then, given $n_0, n_1, \dotsc, n_{k-1}$, let $n_k$ be the largest integer such that 
\[
	n_0 + \frac{n_1}{10} + \dotsb + \frac{n_k}{10^k} \le x.
\]
Then, letting $E$ be the set of the numbers
\[
	n_0 + \frac{n_1}{10} + \dotsb + \frac{n_k}{10^k}, k \in \N,
\]
we have $x = \sup E$. We define $n_0.n_1n_2n_3\dots$ to be the \textbf{decimal expansion} of $x$.
\end{definition}

\section{The Extended Real Number System}
\begin{definition}
The \textbf{extended real number system} consists of $\R$ and two symbols $+\infty$ and $-\infty$, with the normal order and $-\infty < x < \infty$ for all $x \in \R$. 

It is customary to define the following results of operations:
\begin{itemize}
\item $x + \infty = +\infty$, $x - \infty = -\infty$, $\frac{x}{+\infty} = \frac{x}{-\infty} = 0$.
\item If $x > 0$, then $x \cdot (+\infty) = +\infty$.
\item If $x < 0$, then $x \cdot (+\infty) = -\infty$. 
\end{itemize}
\end{definition}

\section{The Complex Field}
\begin{definition}
A \textbf{complex number} is an ordered pair $(a, b)$ of real numbers. We denote $\C$ the set of all complex numbers. Two ordered pairs are equal iff their components are equal. Also, we define
\[
	(a, b) + (c, d) = (a + c, b + d),\ (a, b) \times (c, d) = (ac - bd, ad + bc).
\]
and note that the set of complex numbers forms a field, with $(0, 0)$ and $(1, 0)$ playing the roles of 0 and 1 respectively.

We define $i = (0, 1)$ and notice $i^2 = -1$. This notation is convenient as we can do away with the ordered pair notation and write $(a, b) = a + bi$ instead.

If $z = a + bi$ for $a, b \in \R$, then we define $\Re(z) = a$ and $\Im(z) = b$ to be the \textbf{real part} and \textbf{imaginary part} of $z$ respectively. $\bar{z} = a - bi$ is called the \textbf{conjugate} of $z$. 
\end{definition}

The following properties of complex numbers will be helpful:
\begin{theorem}
If $z, w \in \C$, then 
\begin{enumerate}[(a)]
\item $\bar{z + w} = \bar{z} + \bar{w}$.
\item $\bar{zw} = \bar{z} \cdot \bar{w}$.
\item $z + \bar{z} = \Re(z)$, $z - \bar{z} = 2i\Im(z)$.
\item $z\bar{z} \in \R_{>0}$ when $z \ne 0$.
\end{enumerate}
\begin{proof}
For all of these, we will suppose $z = a + bi$ and $w = c + di$.

For (a), notice that $z + w = (a + c) + (b + d)i$ so
\[
	\bar{z + w} = (a + c) - (b + d)i = (a - bi) + (c - di) = \bar{z} + \bar{w}.
\]

For (b), notice that $zw = (ac - bd) + (ad + bc)i$ and that
\[
	\bar{z} \cdot \bar{w} = (a - bi) \cdot (c - di) = (ac - bd) - (ad + bc)i = \bar{zw}.
\]

For (c), notice that $z + \bar{z} = 2a = 2\Re(z)$ and $z - \bar{z} = 2bi = 2i\Im(z)$.

For (d), notice that $z\bar{z} = a^2 + b^2$, which is real and positive unless $a = b = 0$ (that is, $z = 0$).
\end{proof}
\end{theorem}

Since the value $z\bar{z}$ is a non-negative real number, we can think of it as some measure of how `large' $z$ is: the following definition formalizes this:
\begin{definition}
We define $|z| = \sqrt{z\bar{z}}$, called the \textbf{magnitude} or \textbf{absolute value} of $z$, which exists and is unique from Theorem 3c. When $x \in \R$, $|x| = x$ when $x > 0$ and $-x$ otherwise. In particular, the fact that $x \le |x|$ when $x \in \R$ is sometimes useful.
\end{definition}

The following are helpful properties of the absolute value:
\begin{theorem}
If $z = a + bi, w = c + di \in \C$, then the following are true:
\begin{enumerate}[(a)]
\item $|z| \ge 0$, with equality iff $z = 0$.
\item $|z| = |\bar{z}|$.
\item $|zw| = |z| \cdot |w|$.
\item $|\Re(z)| \le |z|$.
\item $|z + w| \le |z| + |w|$.
\end{enumerate}
\begin{proof}
For (a), clearly $|0| = 0$. Otherwise, $z\bar{z}$ is a positive real number, and thus has a positive square root, which is precisely $|z|$ by definition.

For (b), notice that both sides evaluate to $\sqrt{a^2 + b^2}$.

For (c), if either $z$ or $w$ is 0, the result follows immediately. Otherwise, we can see
\[
	|zw| = \sqrt{zw \bar{zw}} = \sqrt{z\bar{z} w\bar{w}} = \sqrt{z\bar{z}} \sqrt{w\bar{w}} = |z||w|
\]
where the third equality uses Theorem 3(d) since $z\bar{z}$ and $w\bar{w}$ are both positive from Theorem 4(d).

For (d), notice that
\[
	|\Re(z)| = |a| = \sqrt{a^2} \le \sqrt{a^2 + b^2} = |z| 
\]
where the third inequality follows since $f(x) = x^2$ is monotonically increasing as shown in Theorem 3(c). Notice that equality holds iff $b = 0$, that is, $z \in \R$.

For (e), notice that
\[
	|z + w|^2 = (z + w)(\bar{z} + \bar{w}) = z\bar{z} + w\bar{w} + z\bar{w} + w\bar{z} = |z|^2 + |w|^2 + (z\bar{w}) + \bar{(z\bar{w})}.
\]
From Theorems 4(c) and 5(d), we have 
\[
(z\bar{w}) + \bar{(z\bar{w})} = 2\Re(z\bar{w}) \le |2\Re(z\bar{w})| \le 2|z||w|.
\]
Combining these yields
\[
	|z + w|^2 = |z|^2 + |w|^2 + (z\bar{w}) + \bar{(z\bar{w})} \le |z|^2 + |w|^2 + 2|z||w| = (|z| + |w|)^2.
\]
The result follows by taking square roots, since both sides are non-negative real numbers.
\end{proof}
\end{theorem}

Finally, we present a classic inequality, the Cauchy-Schwartz inequality:
\begin{theorem}
If $a_1, \dotsc, a_n, b_1, \dotsc, b_n \in \C$, then $\left| \sum_{j=1}^{n} a_j\bar{b_j} \right|^2 \le \sum_{j=1}^{n} |a_j|^2 \sum_{j=1}^{n} |b_j|^2$. 

\begin{proof}
Let $A = \sum |a_j|^2, B = \sum |b_j|^2 \in \R$ and $C = \sum a_j\bar{b_j}$, so we claim $|C|^2 \le AB$. If $B = 0$, then we must have $b_1 = b_2 = \dotsb = b_n = 0$ and the result follows immediately. Otherwise, if $B > 0$, then
\begin{align*}
0 \le \sum |B a_j - C b_j|^2 &= \sum (B a_j - C b_j)(B \bar{a_j} - \bar{C b_j}) \\
	&= B^2 \sum a_j \bar{a_j} - B\bar{C}\sum a_j \bar{b_j} - BC\sum \bar{a_j}b_j + C\bar{C} \sum b_j\bar{b_j} \\
	&= B^2 \sum|a_j|^2 - B\bar{C}C - BC\bar{C} + C\bar{C} \sum |b_j|^2 \\
	&= B^2 A - B|C|^2 - B|C|^2 + |C|^2 B \\
	&= B(AB - |C|^2)
\end{align*}
and since $B > 0$, we must have $AB - |C|^2 \ge 0$, as required. 
\end{proof}
\end{theorem}

\section{Euclidean Spaces}

Motivated by spaces like $\R^2$ and $\R^3$, we generalize this notion.
\begin{definition}
For $k \in \N$, let $\R^k$ be the set of all ordered $k$-tuples $\textbf{x} = (x_1, x_2, \dotsc, x_k)$, where $x_i \in \R$ are called the \textbf{coordinates} of $\textbf{x}$. We call these tuples \textbf{vectors}.

We define addition as component-wise addition, and scalar multiplication by a real number as component-wise multiplication, so that $\R^k$ is closed under these two operations. The fact that $\R^k$ satisfies these properties, as well as the operations satisfying the associative, commutative, and distributive laws make $\R^k$ into a \textbf{vector space over $\R$}.

We define the \textbf{inner product} of $\textbf{x}$ and $\textbf{y}$ by
\[
	\textbf{x} \cdot \textbf{y} = \sum_{j=1}^{k} x_i y_i
\]
and the \textbf{norm} of $\textbf{x}$ by
\[
	|\textbf{x}| = (\textbf{x} \cdot \textbf{x})^{1/2} = \left(\sum_{j=1}^{k} x_j^2 \right)^{1/2}
\]

Now, $\R^k$ along with its associated inner product and norm is called \textbf{Euclidean $k$-space}.
\end{definition}

We can now use this definition to prove some important properties of $\R^k$:
\begin{theorem}
Suppose $\textbf{x}, \textbf{y}, \textbf{z} \in \R^k$ and $\alpha \in \R$. Then, the following are true:
\begin{enumerate}[(a)]
\item $|\textbf{x}| \ge 0$.
\item $|\textbf{x}| = 0$ iff $\textbf{x} = 0$.
\item $|\alpha \textbf{x}| = |\alpha| |\textbf{x}|$.
\item $|\textbf{x} \cdot \textbf{y}| \le |\textbf{x}| |\textbf{y}|$.
\item $|\textbf{x} + \textbf{y}| \le |\textbf{x}| + |\textbf{y}|$.
\item $|\textbf{x} - \textbf{z}| \le |\textbf{x} - \textbf{y}| + |\textbf{y} - \textbf{z}|$. 
\end{enumerate}
\begin{proof}
The facts (a), (b), and (c) follow directly from the definitions.

For (d), this is precisely the Cauchy-Schwartz inequality restricted to $\R$, applied to the components.

For (e), we write
\begin{align*}
	|\textbf{x} + \textbf{y}|^2 &= (\textbf{x} + \textbf{y})(\textbf{x} + \textbf{y}) \\
		&= \textbf{x} \cdot \textbf{x} + \textbf{x} \cdot \textbf{y} + \textbf{y} \cdot \textbf{x} + \textbf{y} \cdot \textbf{y} \\
		&= |\textbf{x}|^2 + 2 \textbf{x} \cdot \textbf{y} + |\textbf{y}|^2 \\
		&\le |\textbf{x}|^2 + 2 \left|\textbf{x} \cdot \textbf{y}\right| + |\textbf{y}|^2 \\
		&\le |\textbf{x}|^2 + 2 |\textbf{x}| |\textbf{y}| + |\textbf{y}|^2 \\
		&= (|\textbf{x}| + |\textbf{y}|)^2
\end{align*}
and the result follows by taking the square root.

For (f), this is shown by replacing $\textbf{x}$ with $\textbf{x - y}$ and $\textbf{y}$ with $\textbf{y - z}$ in (e).
\end{proof}
\end{theorem}

The fact that $\textbf{x} \ge 0$ with equality iff $\textbf{x} = 0$, and fact (f), called the Triangle Inequality, let us regard $\R^k$ as a \textbf{metric space}. 

$\R^1 = \R$ is called the (real) line, $\R^2$ is the plane. It is noteworthy that the norms in $\R^2$ and $\C$ are consistent.

\section{Appendix: Dedekind's construction of $\R$}

Here we construct $\R$ from $\Q$ in 9 steps.

\begin{enumerate}[Step 1.]
\item \underline{Making $\R$ a set.} 

The elements of $\R$ will be certain subsets of $\Q$, called \textbf{cuts}. A cut is a set $\alpha$ with the following three properties:
\begin{enumerate}[(I)]
\item $\emptyset \subset \alpha \subset \Q$.
\item If $p \in \alpha$ and $q \in \Q$, with $q < p$, then $q \in \alpha$.
\item If $p \in \alpha$, then $p < r$ for some $r \in \alpha$.
\end{enumerate}

We will use $p, q, r, \dotsc$ for rational numbers, and $\alpha, \beta, \gamma, \dotsc$ for cuts. 

Note that condition (II) implies the following useful statements:
\begin{itemize}
\item If $p \in \alpha$ and $q \not\in \alpha$, then $p < q$.
\item If $r \not\in \alpha$ and $r < s$, then $s \not\in \alpha$.
\end{itemize}

\item \underline{Making $\R$ an ordered set.} 

Define an order $<$ on the set of cuts such that $\alpha < \beta$ iff $\alpha \subset \beta$. This is clearly transitive, so it suffices to show that exactly one of $\alpha < \beta$, $\alpha = \beta$, or $\alpha > \beta$ can be true at once. Notice that by the properties of sets, at most one of these can be true at once, so it suffices to show that at least one is.

To do this, suppose that the first two are false. Then, there exists some $q \in \alpha$ which is not in $\beta$. Then, if $p \in \beta$, then we must have $p < q$ from condition (II). However, since $q \in \alpha$ and $p < q$, this also means $p \in \alpha$, so we conclude $\beta \subset \alpha$ so $\alpha > \beta$ as required. Thus, $\R$ is now an ordered set.

\item \underline{Showing $\R$ has the least-upper-bound property.}

Let $A$ be a non-empty subset of $\R$ which is bounded above. Let $\gamma = \bigcup A$. We claim that $\gamma \in \R$ and $\gamma = \sup A$. 

To show $\gamma \in \R$, we prove each condition individually. The first part of condition (I) follows directly from the fact that $A$ is the non-empty union of non-empty subsets of $\Q$. The second follows from the fact that $A$ is bounded above by some $\beta \ne \Q$. For condition (II), pick $p \in \gamma$ and $q \in \Q$. Since $p \in \gamma = \bigcup A$, there exists some $\alpha \in A$ for which $p \in \alpha$. Then, from condition (II) on $\alpha$, we must have $q \in \alpha \subseteq \gamma$, as required. For condition (III), pick the same $p, q, \alpha$. Then, from condition (III) on $\alpha$, there must exist $r \in \alpha \subseteq \gamma$ with $p < r$, as required. Thus, $\gamma \in \R$.

To show that $\gamma = \sup A$, notice that by definition, $\gamma$ is the superset of any $\alpha \in \gamma$, so $\gamma$ is an upper bound for $A$. Let $\beta < \gamma$. Then, there must be some $p$ in $ \gamma$ but not in $\beta$. In particular, there must be some $\alpha \in A$ for which $p \in \alpha$ but not in $\beta$. However, this means $\alpha \not< \beta$ so $\beta$ is not an upper bound for $A$, as required. 

Thus, $\R$ satisfies the least-upper-bound property.

\item \underline{Defining addition on $\R$.}

For $\alpha, \beta \in \R$, we define $\alpha + \beta = \{r + s : r \in \alpha, s \in \beta\}$. For this, we let $0^*$ denote the set of all negative rational numbers. It's easily checked that $0^* \in \R$. We verify the 5 axioms which define $\R$ as a commutative additive group with $+$.

Let $\alpha, \beta, \gamma \in \R$. 
\begin{enumerate}[({A}1)]
\item $\alpha + \beta \in \R$.

Notice that $\alpha + \beta$ is non-empty since $\alpha$ and $\beta$ are non-empty. To show that $\alpha + \beta \ne \Q$, take elements $r' \not\in \alpha, s' \not\in \beta$. Then, $r' + s' > r + s$ for all $r \in \alpha, s \in \beta$, so in particular, $r' + s' \not\in \alpha + \beta$ so $\alpha + \beta \ne \Q$, showing condition (I).

For condition (II) let $p = r + s \in \alpha + \beta$ for some $r \in \alpha, s \in \beta$, and $q < p$. Then, $q - s < p - s = r$ so $q - s \in \alpha$. Thus, $q = (q - s) + s \in \alpha + \beta$. 

For condition (III), let $t > r$ in $\alpha$, so that $t + s > r + s$ in $\alpha + \beta$. 

\item $\alpha + \beta = \beta + \alpha$.

This follows from the definition and the commutativity of $\Q$:
\[
	\alpha + \beta = \{r + s : r \in \alpha, s \in \beta\} = \{s + r : s \in \beta, r \in \alpha\} = \beta + \alpha
\]

\item $(\alpha + \beta) + \gamma = \alpha + (\beta + \gamma)$. 

This follows as above from the definition and the associativity of $\Q$.

\item $\alpha + 0^* = \alpha$.

If $r \in \alpha$, and $s \in 0^*$, then $r + s < r \in \alpha$, so $\alpha + 0^* \subseteq \alpha$. For the other side, take $r \in \alpha$ and $t > r$ in $\alpha$. Then, $r - t < 0$ so $r - t \in 0^*$ and thus $r = t + (r - t) \in \alpha + 0^*$ and thus $\alpha \subseteq \alpha + 0^*$, as required.

\item $-\alpha$ exists.

Let $\beta$ be the set of $p$ such that there exists $r > 0$ such that $-p - r \not\in \alpha$. We claim that $\beta \in \R$ and $\alpha + \beta = 0^*$.

Let $s \not \in \alpha$. Then, $s + 1 \in \beta$ since $(s + 1) - 1 \not\in \alpha$, so $\beta$ is non-empty. Also, if $r \in \alpha$, then $r \not \in \beta$, so $\beta \ne \Q$, satisfying condition (I) for being a cut. For condition (II) let $p \in \beta$ and $q < p$ in $\Q$. Then, $-p - r \not\in \alpha$ for some $r > 0$. Then, $-q - (-q + p + r) = -p - r \not\in \alpha$, and $-q + p + r > r > 0$, so $q \in \beta$, proving condition (II). For condition (III), let $p \in \beta$. Then, there exists $r > 0$ so that $-p - r \not\in \alpha$. Then, $p + r/2 > p$ is in $\beta$ since $-(p + r/2) - (r/2) \not\in \alpha$ and $r/2 > 0$, proving condition (III). So $\beta \in \R$.

Let $p \in \alpha$ and $q \in \beta$. Since $q \in \beta$, there exists $r > 0$ such that $-q - r \not\in \alpha$. Thus, from condition (II) on $\alpha$, $p < -q - r$ and thus $p + q < -r < 0$ so $p + q \in 0^*$. Since our choice of $p$ and $q$ was arbitrary, $\alpha + \beta \subseteq 0^*$. 

Let $t < 0$. We want to find $p \in \alpha$ and $q \in \beta$ such that $p + q = t$. Let $s = -t/2$ so that $s > 0$. Then, by the Archimedean property, there exists an integer $n$ such that $ns \in \alpha$ but $(n + 1)s \not\in \alpha$. Take $p = ns$ and $q = -(n + 2)s = t - p$. Since $-(-(n + 2)s) - s \not\in \alpha$ and $s > 0$, $q \in \beta$, so we are done!

\end{enumerate}

\item \underline{Towards an ordered field}

Notice that $\alpha + \beta \subset \alpha + \gamma$ if $\beta \subset \gamma$, so the first condition for ordered fields is satisfied. It also follows that $\alpha > 0^*$ iff $-\alpha < 0^*$. 

\item \underline{Multiplication, but only somewhat}

Let $\R^+ = \{\alpha \in \R : \alpha > 0^*\}$ be the positive real numbers. If $\alpha, \beta \in \R^+$, define $\alpha\beta$ to be the set of $p \le rs$ in $\R^+$, for some positive $r \in \alpha$, $s \in \beta$ with $r, s > 0$. We define $1^*$ to be the set of all rational numbers less than 1.

It can be shown, similarly to the proofs of (A1) to (A5), that the axioms of multiplication and distributivity hold for multiplication on $\R^+$:

Let $\alpha, \beta > 0^*$ be arbitrary.
\begin{enumerate}[({M}1)]
\item $\alpha \beta \in \R$.

We prove the three conditions separately. For condition (I): choose $r \in \alpha$ and $s \in \beta$ such that $r, s > 0$, which exist since $\alpha, \beta > 0^*$. Then, $rs \le rs$ so $rs \in \alpha\beta$, so $\alpha\beta$ is non-empty. Now, let $p \not\in \alpha$ and $q \not\in \beta$. Then, for any $r \in \alpha$ and $s \in \beta$ with $r, s > 0$, then $r < p$ and $s < q$, so $rs < pq$, implying that $pq \not\in \alpha\beta$, so $\alpha\beta \ne \Q$, proving condition (I).

For condition (II), let $q \in \alpha\beta$ and $p < q$ in $\Q$. Then, $q \le rs$ for $r \in \alpha$, $s \in \beta$ for $r, s > 0$. Then, similarly $p < q \le rs$ so $p \in \alpha\beta$. For condition (III), let $q \in \alpha\beta$ where $q \le rs$ as above. Then, choose $t > r > 0$ and $u > s > 0$ in $\alpha$ and $\beta$ respectively. Then, $tu \le tu$ so $tu \in \alpha\beta$ with $tu > rs$. Thus, $\alpha\beta \in \R$.

\item $\alpha \beta = \beta \alpha$.

This follows directly from the definition, since
\[
	\{rs: r \in \alpha, s \in \beta, r, s > 0\} = \{sr: s \in \beta, r \in \alpha, s, r > 0\}.
\]

\item $(\alpha \beta) \gamma = \alpha (\beta \gamma)$.

We prove that both sets are the set $A$ of all $p \le rst$ for $r \in \alpha$, $s \in \beta$, $t \in \gamma$, where $r, s, t > 0$. Since the right side can be written as $(\beta \gamma) \alpha$ and is thus symmetric, it suffices to prove one side.

Let $p \in A$ so that $p \le rst$, for some $r \in \alpha$, $s \in \beta$, $t \in \gamma$ all positive. Then, $rs \in \alpha\beta$, and $p \le (rs)t$, so $p \in (\alpha \beta) \gamma$. So $A \subseteq (\alpha \beta) \gamma$. The proof of the other side is very similar, so we omit it.

\item $\alpha 1^* = \alpha$.

Let $p \in \alpha 1^*$, so that $p \le rs$ for $r \in \alpha$ with $r > 0$ and $0 < s < 1$. Then, $p \le rs < r$ so $ p \in \alpha$, showing $\alpha 1^* \subseteq \alpha$. Now, let $t \in \alpha$. If $t < 0$, then $2t < t < 0$ so $2t \in \alpha$. Then, $t \le (2t) \cdot (1/2) \in \alpha 1^*$. If $t > 0$, then let $r > t$ in $\alpha$, from condition (III) of a cut. Then, $t/r < 1$, so $t \le r \cdot (t/r) \in \alpha 1^*$, completing this proof.

\item $\alpha^{-1}$ exists.

For $\alpha > 0^*$, denote $\beta$ to be the set of all $p > 0$ such that there exists $r > 0$ with $p^{-1} - r \not\in \alpha$, along with all the non-positive rational numbers. We claim that $\beta \in \R$ and $\alpha\beta = 1^*$.

For condition (I), clearly $0 \in \beta$, so $\beta$ is non-empty. Also, for any positive $q \in \alpha$, $p = \frac{1}{q + 1} > 0$ and $p^{-1} - 1 = q \in \alpha$ so $p \not\in \beta$, and thus $\beta \ne \Q$, proving condition (I) for cuts. 

For condition (II), let $p \in \beta$ and $q < p$. If $q \le 0$, then $q \in 0^* \cup \{0\} \subset\beta$ follows by definition. Otherwise, $p > 0$ so there exists some $r > 0$ such that $p^{-1} - r \not\in \alpha$. Since $q < p$, we have $q^{-1} - r > p^{-1} - r \not\in \alpha$ so $q^{-1} - r \not\in \alpha$ and thus $q \in \beta$, as required. 

Finally for condition (III), let $p \in \beta$. We want to find $q > p$ in $\beta$. If $p < 0$, this is easy since $0 \in \beta$. Otherwise, $p > 0$, so there exists $r > 0$ such that $p^{-1} - r \not\in \alpha$. Then, let $q = (p^{-1} - r/2)^{-1} > p$ (if $p^{-1} = r/2$, choose $r/3$ instead). Then, $q^{-1} - r/2 = p^{-1} - r \not\in \alpha$, so $q \in \beta$.

Now let $t \in \alpha\beta$, so that $t \le rs$ for some positive $r \in \alpha$, $s \in \beta$. Then, $s^{-1} - u \not\in \alpha$ for some $u > 0$. Also, $r - u \in \alpha$, so $r - u < s^{-1} - u$, implying $rs < 1$, so $t \le rs < 1$ and thus $t \in 1^*$. This proves $\alpha\beta \subseteq 1^*$. 

TODO Figure out how to get $1^* \subseteq \alpha\beta$, then show distributivity. 
\end{enumerate}

\item \underline{Multiplication, for real}

Now, define multiplication on all of $\R$ by
\[
	\alpha \beta = 
	\begin{cases}
		0^*	& \text{ if } \alpha = 0^* \text{ or } \beta = 0^* \\
		\alpha \beta & \text{ if } \alpha > 0^*, \beta > 0^* \\
		-(\alpha (-\beta)) & \text{ if } \alpha > 0^*, \beta < 0^* \\
		-((-\alpha) \beta) & \text{ if } \alpha < 0^*, \beta > 0^* \\
		(-\alpha)(-\beta) & \text{ if } \alpha < 0^*, \beta < 0^*
	\end{cases}
\]

The proofs for the multiplication axioms and distributivity follow from Step 6, with repeated use of $\gamma = -(-\gamma)$. The proofs are omitted.

\textbf{\textit{This proves that $\R$ is an ordered field with the least-upper-bound property!}}

\item \underline{Relation to $\Q$}

Associate with each $q \in \Q$, a set $q^* = \{r \in \Q : r < q\}$. Notice that this definition is consistent with our definitions of $0^*$ and $1^*$ above (the asterisk was not a coincidence). These cuts satisfy the following relations:

\begin{enumerate}
\item $r^* + s^* = (r + s)^*$.
\item $r^* s^* = (rs)^*$.
\item $r^* < s^*$ iff $r < s$.
\end{enumerate} 

This shows that $\Q$ is isomorphic to $\Q^*$ whose elements are the rational cuts. This relationship is the reason we can regard $\Q$ as a subfield of $\R$. 
\end{enumerate}

\section{Exercises}
\begin{enumerate}
\item If $r$ is rational and non-zero, and $x$ is irrational, prove that $r + x$ and $rx$ are irrational.

\begin{proof}
Suppose otherwise: then $(r + x) - r$ and $(rx) / r$ would be operations on rational numbers, and thus evaluate to rational numbers. However, these both evaluate to $x$, which is irrational, a contradiction.
\end{proof}

\item Prove that there is no rational number whose square is 12.

\begin{proof}
We first prove that there is no rational number whose square is 3. Suppose that there existed $p / q \in \Q$ whose square is 3, and such that $p$ and $q$ share no factors, cancelling factors when appropriate. Then, multiplying this equality by $q^2$, we have $p^2 = 3q^2$, showing that $p$ is divisible by 3. But, this means $3p'^2 = 3q^2$, which in turn implies $q$ is divisible by 3, contradicting the fact that $p$ and $q$ share no factors.

Now, if some rational number $r$ had $r^2 = 12$, then $r/2$ would be a rational number such that $(r/2)^2 = 3$, contradicting our previous statement.
\end{proof}

\item Prove that in a field $F$, the following statements follow from the axioms of multiplication:
\begin{enumerate}[(a)]
\item If $x \ne 0$ and $xy = xz$ then $y = z$.
\item If $x \ne 0$ and $xy = x$ then $y = 1$.
\item If $x \ne 0$ and $xy = 1$, then $y = 1/x$.
\item If $x \ne 0$ then $1/(1/x) = x$.
\end{enumerate}
\begin{proof}
For (a), if $x \ne 0$, then $1/x \in F$, so we can multiply this to both sides, yielding the result. (b) follows by letting $z = 1$ in (a). (c) follows by letting $z = 1/x$ in (a). (d) follows by letting $x = 1/x$ and $y = 1/(1/x)$ in (c), noticing that $1 = 1/x \cdot x = 1/x \cdot 1/(1/x)$.
\end{proof}

\item Let $E$ be a non-empty subset of an ordered set; suppose $\alpha$ is a lower bound of $E$ and $\beta$ is an upper bound of $E$. Prove that $\alpha \le \beta$.

\begin{proof}
Since $E$ is non-empty, let $x \in E$. Then, since $\alpha$ is a lower bound for $E$, in particular, $\alpha \le x$. Similarly, $x \le \beta$, so $\alpha \le x \le \beta$, as required.
\end{proof}

\item Let $A$ be a non-empty set of real numbers which is bounded below, and $-A$ be the set of all numbers $-x$ for $x \in A$. Prove that $\inf A = -\sup(-A)$. 

\begin{proof}
Since $A$ is non-empty and bounded below, and $\R$ satisfies the least upper bound property, $\alpha = \inf A$ exists. We claim that $-\alpha$ is the least upper bound to $-A$.

First, notice that $-\alpha \ge -x$ for all $-x \in -A$, since $\alpha \le x$ for all $x \in A$. Also, if $-\beta < -\alpha$ were a smaller upper bound for $-A$, then $\beta$ would be a larger lower bound for $A$, contradicting the fact that $\alpha$ is the \textbf{greatest} lower bound for $A$.
\end{proof}

\item Fix $b > 1$.
\begin{enumerate}[(a)]
\item If $m, n, p, q \in \Z$ with $n, q > 0$ and $r = m/n = p/q$, prove that $(b^m)^{1/n} = (b^p)^{1/q}$ so it makes sense to define $b^r = (b^m)^{1/n}$.

\begin{proof}
Let $x = (b^m)^{1/n}$ and $y = (b^p)^{1/q}$. Now, notice that $x^{nq} = (x^n)^q = (b^m)^q = b^{mq}$ and similarly $y^{nq} = b^{np}$. Since $m/n = p/q$, we have $qm = np$, so in fact $x^{nq} = y^{nq}$, so the result follows by the uniqueness of positive real $nq$-th roots.
\end{proof}

\item Prove $b^{r+s} = b^r b^s$ if $r, s \in \Q$. 

\begin{proof}
Write $r = m/n$ and $s = p/q$ for $m, n, p, q \in \Z$ with $n, q > 0$. Then, $r + s = (mq + np) / nq$. Now, notice that from properties of integer powers, $b^{mq + np} = b^{mq} b^{np}$. So,
\[
	(b^{r+s})^{nq} = (b^{(mq + np) / nq})^{nq} = b^{mq + np} = b^{mq} b^{np} = b^{rnq} b^{snq} = (b^r b^s)^{nq}
\]
and the result follows from the uniqueness of $nq$-th positive real roots. 
\end{proof}

\item If $x$ is real, define $B(x)$ to be the set of all numbers $b^t$, where $t$ is rational and $t \le x$. Prove that
\[
	b^r = \sup B(r)
\]
when $r$ is rational. Hence, we can define $b^x = \sup B(x)$ for every real $x$. 

\begin{proof}
We first prove a helpful lemma: if $q > 0$ in $\Q$, then $b^q > 1$. Let $q = m/n$ for $m, n \in \N$ and $n \ne 0$. Since $x^m$ and $x^n$ are monotonically increasing, then if $b > 0$,
\[
	1 < b^{m/n} \iff 1^n = 1 < b^m \iff 1 < b
\]
so $b^q > 1$ as required. It follows that if $r \le t$, then $b^r \le b^t$: simply consider $b^t - b^r = b^r (b^{t-r} - 1) \ge 0$, since both terms are non-negative.

In fact, the previous statement shows that $b^r$ is indeed an upper bound for $B(r)$: if $t \le r$ and thus $b^t \in B(r)$, then $b^t \le b^r$. Since $r \le r$, $b^r \in B(r)$, so any upper bound for $B(r)$ must be at least $b^r$. Thus $b^r$ must be the least upper bound for $B(r)$, as required. 
\end{proof}

\item Prove $b^{x+y} = b^x b^y$ for all $x, y \in \R$.

\begin{proof}
First, we want to show that $b^{x} b^{y}$ is an upper bound for $B(x + y)$. Suppose that $q \le x + y$ in $\Q$. Then, we can write $q = u + v$ for $u \le x, v \le y$ in $\Q$ (from the Archimedean property, we can choose choose $u \in \Q$ in the non-empty interval $[q - y, x]$), so that $b^q = b^{u+v} = b^u b^v$. However, since $u \le x$ and $v \le y$, $b^u \le b^x$ and $b^v \le b^y$, so $b^q = b^u b^v \le b^x b^y$ as required.

Now, suppose $r < b^x b^y$. We need to find some $q \le x + y$ in $\Q$ such that $b^q > r$, to show that $r$ is not an upper bound for $B(x + y)$. 

Dividing both sides by $b^x$ yields $r/b^x < b^y = \sup B(y)$, so there exists some $v \le y$ in $\Q$ such that $b^v > r/b^x$ so $r < b^x b^v$. Dividing this new inequality by $b^v$ yields $r / b^v < b^x = \sup B(x)$ so there exists $u \le x$ in $\Q$ such that $b^u > r / b^v$. Thus, $r < b^u b^v = b^{u + v}$ with $q = u + v \le x + y$, as required. 
\end{proof}
\end{enumerate}

\item Fix $b > 1, y > 0$, and prove that there is a unique real $x$ such that $b^x = y$, by completing the following outline. (This $x$ is called the \textbf{logarithm of $y$ to the base $b$}.)

\begin{enumerate}[(a)]
\item For any positive integer $n$, we have $b^n - 1 \ge n(b - 1)$. 
\begin{proof}
Factor $b^n - 1 = (b - 1)(b^{n - 1} + b^{n - 2} + \dotsb + b + 1)$. Since $b^k > 1$ for every positive integer $k$ (inductively), the above expression is greater than or equal to $(b - 1)(1 + 1 + \dotsb + 1) = n(b - 1)$, where there are $n$ 1's, as required.
\end{proof}

\item Hence $b - 1 \ge n(b^{1/n} - 1)$.
\begin{proof}
Since $f(x) = x^n$ is monotonically increasing, $b^{1/n} > 1$ since $b > 1$, so substituting $b^{1/n}$ for $b$ in (a) yields the result.
\end{proof}

\item If $t > 1$ and $n > (b - 1)/(t - 1)$, then $b^{1/n} < t$.
\begin{proof}
We have 
\[
	n > \frac{b - 1}{t - 1} \ge \frac{n(b^{1/n} - 1)}{t - 1}
\]
so cancelling the positive $n$ from the outermost expressions, multiplying by the positive value $t - 1$, and adding 1 to both sides yields the result.
\end{proof}

\item If $w$ is such that $b^w < y$, then $b^{w + 1/n} < y$ for sufficiently large $n$; to see this, apply part (c) with $t = y \cdot b^{-w}$.
\begin{proof}
Following the prompt, we notice that letting $t = y \cdot b^{-w} > 1$ in (c), if $n > (b - 1)/(y \cdot b^{-w} - 1) \in \R$, then $b^{1/n} < y \cdot b^{-w}$, which after multiplying both sides by the positive value $b^w$, gives the result.
\end{proof}

\item If $b^w > y$, then $b^{w - 1/n} > y$ for sufficiently large $n$.
\begin{proof}
Similarly, letting $t = b^w / y$ in (c) yields $b^{1/n} < b^w / y$ for $n > (b - 1)/(b^w / y - 1) \in \R$, after which rearranging for $y$ yields the result. 
\end{proof}

\item Let $A$ be the set of all $w$ such that $b^w < y$, and show that $x = \sup A$ satisfies $b^x = y$.
\begin{proof}
Suppose otherwise. If $b^x < y$, then from (d), there exists some $n \in \N$ such that $b^{x + 1/n} < y$, so $x + 1/n \in A$, contradicting the fact that $x$ is an upper bound for $A$. 

Similarly, if $b^x > y$, then from (e), there exists some $n \in \N$ such that $b^{x - 1/n} > y$, and so for any $w \in A$, we have $b^w < y < b^{x - 1/n}$ and thus $w < x - 1/n$ from the lemma in Exercise 6(c), making $x - 1/n$ a smaller upper bound for $A$, a contradiction.
\end{proof}

\item Prove that this $x$ is unique.
\begin{proof}
Suppose that $x \ne y$ in $\R$ and $b^x = b^y$. Without loss of generality, suppose $x < y$. But then $b^x < b^y$ from the statement proved in Exercise 6(c), contradicting our assumption, so we are done.
\end{proof}
\end{enumerate}

\item Prove that no order can be defined in the complex field that turns it into an ordered field. Hint: $-1$ is a square.

\begin{proof}
Suppose otherwise. Then, notice that since $1 = 1^2$ and $-1 = i^2$ are both non-zero squares, they are both positive. Thus, their sum, 0, would have to be positive, a contradiction.
\end{proof}

\item Suppose $z = a + bi$ and $w = c + di$. Define $z < w$ if $a < c$,  and also if $a = c$ but $b < d$. Prove that this turns the set of all complex numbers into an ordered set. (This type of order relation is called a \textbf{dictionary order} or \textbf{lexicographic order}, for obvious reasons.) Does this ordered set have the least-upper-bound property?

\begin{proof}
First, we show that exactly one of $z < w$, $z = w$, or $z > w$ is true. Suppose that $z \ne w$, so that $a \ne c$ or $b \ne d$. If $a \ne c$, then either $a < c$, in which case $z < w$, or $a > c$, in which case $z > w$. Otherwise, either $b < d$, in which case $z < w$, or $b > d$, in which case $z > w$. 

Next, we show that this order is transitive. Suppose $a + bi < c + di$ and $c + di < e + fi$. If $a < c$, then since $c \le e$, $a < e$ and thus $a + bi < e + fi$. If $a = c$, then $b < d$. If further $c = e$, then $b < d < f$ and thus $a + bi < e + fi$. If further $c < e$, then $a = c < e$ so $a + bi < e + fi$. In all cases, $a + bi < e + fi$.

This ordered set does not have the least-upper-bound property: consider the set $A = i\R = \{ir : r \in \R\}$, which is bounded above by $1$. We claim that though $A$ is non-empty and bounded above, there does not exist a least upper bound. Let $\alpha$ be an upper bound for $A$. Notice that $\alpha$ must have a non-negative real component. If $\alpha$ has a strictly positive real component, say $r$, then $r/2$ will be smaller than $r$ but still larger than all of $A$. Thus, a least upper bound will never have positive real component. If $\alpha$ had real component 0, then $\alpha + 1 \in A$ would be larger than it, a contradiction. Thus, no least upper bound for $A$ exists.
\end{proof}

\item Suppose $z = a + bi$, $w = u + vi$, and
\[
	a = \left(\frac{|w| + u}{2}\right)^{1/2},\ b = \left(\frac{|w| - u}{2}\right)^{1/2}.
\]
Prove that $z^2 = w$ if $v \ge 0$ and that $(\bar{z})^2 = w$ if $v \le 0$. Conclude that every complex number (with one exception!) has two complex square roots.

\begin{proof}
Notice that $(a + bi)^2 = (a^2 - b^2) + i(2ab)$, so
\[
	z^2 =  \left(\frac{|w| + u}{2} - \frac{|w| - u}{2} \right) + 2i \left( \sqrt{\frac{|w| + u}{2}} \cdot \sqrt{\frac{|w| - u}{2}} \right) = u + i \sqrt{|w|^2 - u^2} = u + i|v|
\]
so $z^2 = u + iv$ when $v \ge 0$. Similarly,
\[
	\bar{z}^2 = (a^2 - b^2) - 2iab = u - i|v| = u + iv
\]
if $v < 0$. Thus, if $v \ge 0$, then $z$ and $-z$ are two complex square roots of $w$, and similarly $\pm \bar{z}$ are square roots of $w$ if $v < 0$. The exception here is of course when $z = -z$, namely when $z = 0$ and thus $w = z^2 = 0$.
\end{proof}

\item If $z$ is a complex number, prove that there exists an $r \ge 0$ and a complex number $w$ with $|w| = 1$ such that $z = rw$. Are $w$ and $r$ always uniquely determined by $z$?

\begin{proof}
Let $r = |z| \ge 0$ and $w = z / |z|$. Notice that $|w| = |z| / ||z|| = 1$, and $z = rw$ by definition. $w$ and $r$ are uniquely defined for all $z \in \C$ except 0, in which case $r = 0$ and $w$ can be any complex number with unit absolute value.

Suppose $z \ne 0$ and $z = r_1 w_1 = r_2 w_2$ with the required conditions. Take absolute values to get $|z| = |r_1||w_1| = |r_2||w_2|$, which implies $|r_1| = |r_2|$. Since $r_1, r_2 \ge 0$, we must have $r_1 = r_2$, uniquely determining $w_1$ and $w_2$, as required.
\end{proof}

\item If $z_1, \dotsc, z_n$ are complex, prove that
\[
	|z_1 + z_2 + \dotsb + z_n| \le |z_1| + |z_2| + \dotsb + |z_n|.
\]

\begin{proof}
We proceed by induction on $n$. The base case is given by Theorem 7(e), as applied to complex numbers taken as ordered pairs of real numbers. Then, for $n > 2$, assuming the statement is true for $n - 1$ inductively, we have
\begin{align*}
	|z_1 + z_2 + \dotsb + z_{n-1} + z_n| &\le |z_1 + z_2 + \dotsb + z_{n-1}| + |z_n| \\
		&\le |z_1| + |z_2| + \dotsb + |z_{n-1}| + |z_n|,
\end{align*}
completing the induction.
\end{proof}

\item If $x, y \in \C$, prove that 
\[
	||x| - |y|| \le |x - y|.
\]

\begin{proof}
Notice $|x| = |(y - x) - y| \le |y - x| + |y|$ so $|x - y| = |y - x| \ge |x| - |y|$. 

Similarly, $|y| = |(y - x) + x| \le |y - x| + |x|$ so $|x - y| \ge |y| - |x|$. Combining these two inequalities yields
\[
	-|x - y| \le |x| - |y| \le |x - y| 
\]
and so
\[
	||x| - |y|| \le |x - y|,
\]
as required.
\end{proof}

\item If $z \in \C$ such that $|z| = 1$, compute
\[
	|1 + z|^2 + |1 - z|^2
\]

\begin{proof}
Simply write
\[
	|1 + z|^2 + |1 - z|^2 = (1 + z)(1 + \bar{z}) + (1 - z)(1 - \bar{z}) = 1 + z + \bar{z} + z\bar{z} + 1 - z - \bar{z} + z\bar{z} = 4.
\]
\end{proof}

\item Under what conditions does equality hold in the Schwartz inequality?

\begin{proof}
Notice that in the proof, equality holds iff $\sum |Ba_j - Cb_j|^2 = 0$, and equivalently $a_j = Cb_j/B$ for each $j$. In particular, if the $a_j$ are each the same multiple of $b_j$, then equality will hold.
\end{proof}

\item Suppose $k \ge 3$, $\textbf{x}, \textbf{y} \in \R^k$, $|\textbf{x} - \textbf{y}| = d > 0$, and $r > 0$. Prove:
\begin{enumerate}[(a)]
\item If $2r > d$, there are infinitely many $\textbf{z} \in \R^k$ such that 
\[
	|\textbf{z} - \textbf{x}| = |\textbf{z} - \textbf{y}| = r.
\]

\begin{proof} Let $\textbf{m} = \frac{1}{2} ( \textbf{x} + \textbf{y} )$. Without loss of generality, suppose $\textbf{y} = -\textbf{x}$, translating when necessary. Also, let $t = \sqrt{r^2 - d^2/4}$, which is well defined since $2r > d$. Then, the set of vectors on the hyperplane orthogonal to the line crossing $\textbf{x}$ and $\textbf{y}$ with distance $t$ to $\textbf{m}$ satisfy the required equation, and there are infinitely many.
\end{proof}

\item If $2r = d$, there is exactly one such $\textbf{z}$.

\begin{proof}
If such a $\textbf{z}$ existed, then $d = |\textbf{x} - \textbf{y}| \le |\textbf{x} - \textbf{z}| + |\textbf{z} - \textbf{y}| = 2r = d$, so equality must hold in the triangle inequality. This happens when the three points are co-linear, and there is a unique point which satisfies this, namely $\textbf{z} = \textbf{m}$ as defined above.
\end{proof}

\item If $2r < d$, there are no such $\textbf{z}$.
\begin{proof}
If such a $\textbf{z}$ existed, then $d = |\textbf{x} - \textbf{y}| \le |\textbf{x} - \textbf{z}| + |\textbf{z} - \textbf{y}| = 2r < d$, a contradiction.
\end{proof}
\end{enumerate}

How must these statements be modified if $k < 3$?

\begin{proof}
If $k = 2$, then (a) will have 2 solutions instead, and everything else remains the same. If $k = 1$, (a) will have no solutions.
\end{proof}

\item Prove that
\[
	|\textbf{x} + \textbf{y}|^2 + |\textbf{x} - \textbf{y}|^2 = 2 |\textbf{x}|^2 + 2 |\textbf{y}|^2
\]
if $x, y \in \R^k$. Interpret this geometrically, as a statement about parallelograms.

\begin{proof}
We expand:
\begin{align*}
|\textbf{x} + \textbf{y}|^2 + |\textbf{x} - \textbf{y}|^2 &= (\textbf{x} + \textbf{y}) \cdot (\textbf{x} + \textbf{y}) + (\textbf{x} - \textbf{y}) \cdot (\textbf{x} - \textbf{y}) \\
	&= \textbf{x} \cdot \textbf{x} + \textbf{x} \cdot \textbf{y} + \textbf{y} \cdot \textbf{x} + \textbf{y} \cdot \textbf{y} + \textbf{x} \cdot \textbf{x} - \textbf{x} \cdot \textbf{y} - \textbf{y} \cdot \textbf{x} + \textbf{y} \cdot \textbf{y} \\
	&= 2 |\textbf{x}|^2 + 2 |\textbf{y}|^2
\end{align*}

The interpretation of this is that in a parallelogram, the sums of the squares of the lengths of the diagonals is equal to the sum of the squares of the lengths of the four sides.
\end{proof}

\item If $k \ge 2$ and $\textbf{x} \in \R^k$, prove that there exists $\textbf{y} \in \R^k$ such that $\textbf{y} \ne 0$ but $\textbf{x} \cdot \textbf{y} = 0$. Is this also true if $k = 1$?

\begin{proof}
If $\textbf{x} = 0$, then any non-zero vector will do. If $\textbf{x}$ has only one non-zero component, then choose $i$ such that $x_i = 0$ and let $\textbf{y}$ be the vector with $y_i = 1$ and $y_j = 0$ if $j \ne i$. Otherwise, if $\textbf{x} = (x_1, \dotsc, x_k)$, there exists $x_i \ne 0$. Then, let $y_j = x_j$ if $j \ne i$, and $y_i = \frac{x_1^2 + x_2^2 + \dotsc + x_k^2 - x_i^2}{x_i}$. Then, 
\[
	\textbf{x} \cdot \textbf{y} = \sum_{j=1}^{k} x_j y_j = \left(\sum_{j=1}^{k} x_j x_j\right) - x_i^2 - x_iy_i = 0
\]
as required. Notice that $\textbf{y}$ is non-zero since there exists $y_k = x_k \ne 0$ for $k \ne i$ by presumption.

This statement is not true for $k = 1$: consider $\textbf{x} = (1)$.
\end{proof}

\item Suppose $\textbf{a}, \textbf{b} \in \R^k$. Find $\textbf{c} \in \R^k$ and $r > 0$ such that
\[
	|\textbf{x} - \textbf{a}| = 2 |\textbf{x} - \textbf{b}|
\]
iff $|\textbf{x} - \textbf{c}| = r$.

\begin{proof}
This is an example of \textbf{Circles of Apollonius}. $\textbf{c} = \frac{1}{3} \left(4\textbf{b} - \textbf{a}\right)$, $r = \frac{2}{3} |\textbf{b} - \textbf{a}|$. 
\end{proof}

\item With reference to the Appendix, suppose that property (III) were omitted from the definition of a cut. Keep the same definitions of order and addition. Show that the resulting ordered set has the least-upper-bound property, that addition satisfies axioms (A1) to (A4) (with a slightly different zero-element!) but that (A5) fails.

\begin{proof}
TODO
\end{proof}
\end{enumerate}